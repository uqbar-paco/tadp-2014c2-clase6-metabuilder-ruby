\documentclass[spanish,a4paper]{article}
\usepackage{verbatim}
\usepackage{listings}
\usepackage{multicol}
\newcommand{\tipoEvaluacion}{Clase 6 -- Ejercicio de Metaprogramación}
\newcommand{\version}{final} % final, interna o con soluciones (ojo escribirlo 'tal cual').
\newcommand{\cuatrimestre}{2º Cuatrimestre 2014} 
\usepackage[utf8]{inputenc}
\usepackage{amstext}
\usepackage{ifthen}
\usepackage{graphicx}
\usepackage{array}
\usepackage{xspace}
\usepackage{color}
\usepackage{fancyvrb}
\usepackage{version}
\usepackage{theorem}
\usepackage{listings}

\usepackage[spanish]{babel}

\textwidth 17.5cm
\textheight 26cm
\oddsidemargin -1cm
%\evensidemargin -.3cm
\topmargin -1cm
\headheight 0cm
\headsep 0cm

%\textwidth 16cm
%\textheight 25cm
%\oddsidemargin -.64cm
%\evensidemargin -.64cm
%\topmargin -1.5cm
%\headheight 0cm
%\headsep 0cm

\definecolor{dkgreen}{rgb}{0,0.6,0}
\definecolor{gray}{rgb}{0.5,0.5,0.5}
\definecolor{mauve}{rgb}{0.58,0,0.82}

\lstloadlanguages{Ruby}
\lstdefinelanguage{smalltalk}{
  morekeywords={self,super},
  otherkeywords={:=,^ },
  sensitive=true,
  morecomment=[n]{"},
  morestring=[b]'
}

% Default settings for code listings
\lstset{%frame=tb,
  aboveskip=3mm,
  belowskip=3mm,
  showstringspaces=false,
  columns=flexible,
  basicstyle={\footnotesize\ttfamily},
  numbers=none,
  numberstyle=\tiny\color{gray},
  keywordstyle=\color{blue},
  commentstyle=\color{dkgreen},
  stringstyle=\color{mauve},
  %frame=single,
  breaklines=true,
  breakatwhitespace=true
  tabsize=3
}

\renewcommand{\thepage}{}
\newcommand{\ignore}[1]{}
\newcommand{\mat}[1]{\ensuremath{#1}}

\newenvironment{display}
   {\begin{list}{}{\setlength{\topsep}{0cm}
                   \setlength{\leftmargin}{0cm}
                  }}{\end{list}}

\newcommand{\textfol}[1]{\ensuremath{\textsf{#1}}}
\newcommand{\textcode}[1]{{\normalfont{\texttt{#1}}}}
\newcommand{\propername}[1]{\textsc{#1}}

%\renewcommand{\labelenumi}{\textbf{\alph{enumi})}}
\renewcommand{\labelenumii}{\textbf{\alph{enumii})}}

\newcommand{\pred}[2]{\ensuremath{\text{\textbf{#1}}(#2)}}
\newcommand{\predName}[1]{\ensuremath{\text{\textbf{#1}}}}
\newcommand{\var}[1]{\ensuremath{#1}}
\newcommand{\code}[1]{\texttt{\small #1}}

\newcommand{\paratodo}[2]{\ensuremath{(\forall{#1})#2}}
\newcommand{\existe}[2]{\ensuremath{(\exists{#1})#2}}
\newcommand{\implicasp}{\ensuremath{\Rightarrow}}
\newcommand{\ysp}{\ensuremath{\,\wedge\,}}
\newcommand{\osp}{\ensuremath{\,\vee\,}}
\newcommand{\nosp}{\ensuremath{\neg}}
\newcommand{\implica}{\ensuremath{\Rightarrow}}
\newcommand{\y}{\ensuremath{\,\wedge\,}}
\renewcommand{\o}{\ensuremath{\,\vee\,}}
\newcommand{\no}{\ensuremath{\neg}}

\newcommand{\agujero}[1]{\ensuremath{<}\textit{#1}\ensuremath{>}}
\newcommand{\vble}[1]{\ensuremath{#1}}
\newcommand{\mkP}[2]{\ensuremath{p_{{#1#2}}}}

%%%%%%%%%%%%%%%%%%%%%%%%%%%%%%%%%%%%%%%%%%%%%%%%%%%
% Manejo de versiones (final o interna)
%%%%%%%%%%%%%%%%%%%%%%%%%%%%%%%%%%%%%%%%%%%%%%%%%%%

\newcommand{\final}[1]{\ifthenelse{\equal{\version}{interna}}{}{#1}}
\newcommand{\nofinal}[1]{\ifthenelse{\equal{\version}{interna}}{#1}{}}

\newcommand{\consolu}[1]{\ifthenelse{\equal{\version}{con soluciones}}{#1}{}}
\newcommand{\sinsolu}[1]{\ifthenelse{\equal{\version}{con soluciones}}{}{#1}}

\newboolean{final}
\final{\setboolean{final}{true}}
\nofinal{\setboolean{final}{false}}

\newcommand{\internal}[1]{ 
 \nofinal{\begin{flushleft} \textcolor{blue}{\upshape #1} \end{flushleft}}
}

\newcommand{\comentario}[1]{\textbf{\internal{#1}}}
\newcommand{\objetivo}[1]{\internal{\textbf{Objetivo:} #1}}

\sinsolu{\excludeversion{solucion}}
\consolu{
	\newcommand{\solucionTitle}{{\normalfont\textbf{\textcolor{blue}{Solución posible:}}}}
	\DefineVerbatimEnvironment{solucion}{Verbatim}
		{fontshape=n,tabsize=0,fontsize=\small,xleftmargin=16pt,formatcom=\solucionTitle\color{blue}}
}

\DefineVerbatimEnvironment{ejemplo}{Verbatim}{fontshape=n,tabsize=0,fontsize=\small,xleftmargin=16pt,}

%%%%%%%%%%%%%%%%%%%%%%%%%%%%%%%%%%%%%%%%%%%%%%%%%%%
% Cuestiones específicas de cada paradigma
%%%%%%%%%%%%%%%%%%%%%%%%%%%%%%%%%%%%%%%%%%%%%%%%%%%

\newcommand{\paradigma}{tadp} % medio chancho ya lo voy a ordenar.
\newcommand{\funcional}[1]{\ifthenelse{\equal{\paradigma}{Funcional}}{#1}{}}
\newcommand{\logico}[1]{\ifthenelse{\equal{\paradigma}{L\'ogico}}{#1}{}}
\newcommand{\objetos}[1]{\ifthenelse{\equal{\paradigma}{Objetos}}{#1}{}}

\newcommand{\aclaracionDocentes}{
	\internal{
	\textit{
	\small\textbf{Recordatorio, qué queremos evaluar:}
	\footnotesize
	\funcional{
		\begin{itemize}
			\item 
				Funciones sobre listas, algunas de: map, filter, any, all, find. 
				Y sí o sí algún fold.
			\item Alguna función de orden superior que hagan ellos.
			\item Tuplas, pattern matching, composición y aplicación parcial se van mechando con lo anterior
			\item Y alguna recursiva podría ser.
			\item Guardas, lambda, listas por comprensión y definiciones locales los considero opcionales.			
			\item Además estaría bueno mechar una mayoría de funciones ``directas'', 
						es decir, donde se aplica más bien un concepto y ya, 
						con otras más elaboradas en las que tengan que pensar más ``estratégicamente''.
		\end{itemize}
	}
	\logico{
		\begin{itemize}
			\item Básicos: Manejos de predicados e individuos, and y or.
			\item Functores y polimorfismo entre functores.
			\item Existe y paratodo.
			\item Generación.
			\item (En menor medida) listas y findall.
			\item (Opcional) recursividad.
		\end{itemize}
	}
	\objetos{
		\begin{itemize}
			\item Básicos: Objeto, mensaje, parámetro. Referencia. 
			\item Polimorfismo y delegación.
			\item Herencia, redefinición y super.
			\item Variables locales, de instancia y de clase.
			\item Manejo de colecciones: collect, select, ordenar, inject into. 
						En menor medida diferenciar entre Set, OrderedCollection y SortedCollection.
						También puede ser interesante tener operaciones sobre conjuntos.
		\end{itemize}
	}
	\small\textbf{¿Están de acuerdo con los objetivos propuestos?}
	}}}

\newcommand{\aclaracionAlumnos}{
	\textit{
    \small % \normalsize
    \textbf{Aclaraciones:}
    \footnotesize  % \small
	\begin{itemize}
		\item 
			Esta evaluación es a libro abierto, 
			pueden usar todo lo que tengan en la carpeta y los apuntes que deseen.
      	\item 
			Es muy importante poner nombre, nro.\ de legajo, nro.\ de hoja y 
			cantidad total de hojas en cada hoja.
		\item 
			Recuerden que la intención es medir cuánto se sabe de los temas de la materia;
			en la solución deben mostrar su conocimiento de los conceptos y herramientas que aprendieron.
			\funcional{De estas ideas las que más nos interesan son: aplicación parcial, composición y orden superior.}
	      \end{itemize}
   }
}
\newcommand{\aclaracion}{
  \fbox {
    \parbox{.95\textwidth}
    {\aclaracionAlumnos \aclaracionDocentes}
  }
}

\newcommand{\tipoevaluacion}{Ejercicio Metaprogramación: Metabuilder}

\title{Técnicas Avanzadas de Programación -- UTN -- FRBA
       \\ \cuatrimestre  \
       \\ \tipoevaluacion\
       \\ \nofinal{\large \medskip Versión \version\\}
      }
%\date{\aclaracion}
\date{\vspace{-5ex}}

\newcommand{\newest}[1]{#1} %\textbf{Cursadas$\geq$2000:} #1 }
\newcommand{\oldest}[1]{}%{#1}

\newcommand{\flecha}{->}
\newcommand{\newconcept}[1]{\emph{#1}}

\DefineShortVerb{\|}

\author{}

%%%%%%%%%%%%%%%%%%%%%%%%%%%%%%%%%%%%%%%%%%%%%%%%%%%
% Environments para ejercicios
%%%%%%%%%%%%%%%%%%%%%%%%%%%%%%%%%%%%%%%%%%%%%%%%%%%

\newcounter{ejercicioNumero}
\addtocounter{ejercicioNumero}{1}
\newenvironment{ejercicio}
	{\vspace{2mm} \noindent \textbf{Ejercicio \arabic{ejercicioNumero}} \addtocounter{ejercicioNumero}{1} \\*[3mm]}
	{}
	
\newenvironment{nota}
	{\noindent \textbf{Nota}:\\}
	{}
	
	
	
\lstset{ %
  language=Ruby,                % choose the language of the code
  basicstyle={\footnotesize\ttfamily},
  backgroundcolor=\color{white},  % choose the background color. You must add \usepackage{color}
  showspaces=false,               % show spaces adding particular underscores
  showstringspaces=false,         % underline spaces within strings
  showtabs=false,                 % show tabs within strings adding particular underscores
  frame=single,	                % adds a frame around the code
  tabsize=2,	                % sets default tabsize to 2 spaces
  captionpos=true,                   % sets the caption-position to bottom
  breaklines=true,                % sets automatic line breaking
  breakatwhitespace=true,        % sets if automatic breaks should only happen at whitespace
  escapeinside={\%*}{*)/},         % if you want to add a comment within your code
  morekeywords={*,...}            % if you want to add more keywords to the set
}



\begin{document}
\maketitle

\section*{Dominio}
Se desea poder crear un builder para crear builders (o sea, un metabuilder). La idea general para todos los puntos es poder indicarle a un objeto builder como se debería comportar otro builder genérico y luego obtener una instancia del mismo para aplicarlo en el dominio particular que corresponda.
El metabuilder, entonces, será nuestro framework de builders que es agnóstico de los dominios de negocio, mientras que los builders generados por este estarán particularizados para un dominio.

Este ejercicio fue un parcial y se utilizará como ejercicio de preparación para el parcial de lengujes dinámicos.

\section{El metabuilder}

Se desea poder crear un metabuilder al cual se le defina cuales son las propiedades del objeto a construir y la clase del mismo.
Es importante destacar que el builder de dominio sólo debe permitir configurar las propiedades declaradas. Cualquier otra propiedad que se le quiera configurar se debe considerar como un error y debe tirar una excepción en algún momento.

  \begin{lstlisting}[language=Ruby]
class Perro
  attr_accessor :raza, :edad, :duenio

  def initialize
    @duenio = "Alguien"
  end
end
  \end{lstlisting}
  
  \begin{lstlisting}[language=Ruby]
it 'puedo crear un builder de perros' do
  metabuilder = Metabuilder.new.
      set_target_class(Perro).
      add_property(:raza).
      add_property(:edad)

  builder_de_perros = metabuilder.build
  builder_de_perros.raza = "Fox Terrier"
  builder_de_perros.edad = 4
  perro = builder_de_perros.build

  expect(perro.raza).to eq("Fox Terrier")
  expect(perro.edad).to eq(4)
  expect(perro.duenio).to eq("Alguien")
end
  \end{lstlisting}

\section{Sintaxis del metabuilder}
Considerando el punto 1, se desea extender el Metabuilder para que soporte la siguiente sintaxis (la sintaxis anterior debe seguir siendo soportada).

  \begin{lstlisting}[language=Ruby]
it 'soporta configuracion por bloques' do
  metabuilder = Metabuilder.config {
    target_class Perro
    property :raza
    property :edad
  }

  builder_de_perros = metabuilder.build
  builder_de_perros.raza = "Fox Terrier"
  builder_de_perros.edad = 4
  perro = builder_de_perros.build

  expect(perro.raza).to eq("Fox Terrier")
  expect(perro.edad).to eq(4)
end
  \end{lstlisting}

\section{Validaciones}

Se desean incorporar validaciones, las cuales se ejecutan en el momento de crear la instancia. Si no se cumplen todas las validaciones, entonces no se debe de crear la instancia y el builder debe tirar una excepción.
Tener en cuenta que dentro de las validaciones se debe poder hacer referencia a cualquiera de las propiedades definidas, simplemente escribiendo el nombre de las mismas.

  \begin{lstlisting}[language=Ruby]
it 'puedo definir validaciones que rompen' do
  metabuilder = Metabuilder.config {
    target_class Perro
    property :raza
    property :edad
    validate {
      ["Fox Terrier", "San Bernardo"].include? raza
    }
    validate {
      edad > 0 && edad < 20
    }
  }

  builder_de_perros = metabuilder.build
  builder_de_perros.raza = "Fox Terrier"
  builder_de_perros.edad = -5
  expect {
    builder_de_perros.build
  }.to raise_error ValidationError
end
  \end{lstlisting}

\section{Comportamiento}

Para este punto se quiere poder agregar ciertos comportamientos según una condición dada. La idea es poder agregar dichos comportamientos solamente a la instancia construida, si cumple la condición.

  \begin{lstlisting}[language=Ruby]
it 'agrega metodos cuando se cumple la condicion' do
  metabuilder = Metabuilder.config {
    target_class Perro
    property :raza
    property :edad
    conditional_method :caza_un_zorro, proc { raza == "Fox Terrier" && edad > 2 }, proc {
      "Ahora voy"
    }
  }

  builder1 = metabuilder.build
  builder1.raza = "Fox Terrier"
  builder1.edad = 3
  fox_terrier = builder1.build

  expect(fox_terrier.caza_un_zorro).to eq("Ahora voy")

  builder2 = metabuilder.build
  builder2.raza = "San Bernardo"
  builder2.edad = 3
  san_bernardo = builder2.build

  expect {
    san_bernardo.caza_un_zorro
  }.to raise_error(NoMethodError)
end
  \end{lstlisting}
\end{document}
